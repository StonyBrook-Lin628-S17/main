\chapter{The Problems of Finite-State Automata for Syntax}
\label{cha:REGsyntax}

For Tuesday, do 3.1, 3.2, and 3.4. The rest is optional.

\begin{exercise}
    Give a finite-state automaton that correctly handles Principle A for sentences of the form below:
    %
    \begin{exe}
        \ex
        \begin{xlist}
            \ex[] {Mary likes herself.}
            \ex[] {John likes himself.}
            \ex[*] {The old woman likes himself.}
            \ex[*] {The old, ugly fisherman likes herself.}
        \end{xlist}
        \ex
        \begin{xlist}
            \ex {John introduced himself\slash$^*$herself to himself\slash$^*$herself.}
            \ex {John introduced himself\slash$^*$herself to him\slash her.}
            \ex {John introduced him\slash her to himself\slash$^*$herself.}
        \end{xlist}
        \ex
        \begin{xlist}
            \ex {Mary thinks that Bill\slash Sue thinks that John likes himself\slash$^*$herself.}
            \ex {Mary seems to Bill to appear to John to like herself\slash$^*$himself.}
        \end{xlist}
    \end{exe}
\end{exercise}

\begin{exercise}
    How does your automaton need to be modified to also handle this sentence:
    %
    \begin{exe}
        \ex The man that Mary says that she seems to like himself\slash$^*$herself.
    \end{exe}
\end{exercise}

\begin{exercise}
    Bavarian German has inflected complementizers.
    This means that certain complementizers (sometimes optionally) agree in number and person with the subject of their clause.
    %
    \begin{exe}
        \ex
        \begin{xlist}
            \ex {\gll I sags da wann-st (du) dran bist.\\
                      I tell.it you.\g{sg} when(-\g{2.sg}) (you) next are\\}
            \ex {\gll I sags da wann(-st) du dran bist.\\
                      I tell.it you.\g{sg} when(-\g{2.sg}) (you) next are\\}
            \ex {\gll I sags euch wann-ts (es) dran seids.\\
                      I tell.it you.\g{pl} when(-\g{2.pl}) (you) next are\\}
            \ex {\gll I sags euch wann($^*$-ts) ihr dran seids.\\
                      I tell.it you.\g{pl} when(-\g{2.pl}) you next are\\
                 \glt `I'll tell you when you're next.'}
        \end{xlist}
    \end{exe}
    %
    Other inflected complementizers include \emph{wenn} `if` and for some speakers \emph{dass} `that`.
    No other person-number combinations show inflection, only second person singular and plural does.
    For second person plural, the subject must be \emph{es}, not the semantically equivalent \emph{ihr}.

    Interesting things happen with coordinated subjects:
    %
    \begin{exe}
        \ex
        \begin{xlist}
            \ex {\gll \ldots wenn(-st)\slash($^*$-ts) [du und die Maria] woll-ts\slash-en.\\
                      \ldots if-\g{2.sg}\slash$^*$-\g{2.pl} [you.\g{sg} and the Mary] want-\g{2.pl}\slash-\g{3.pl}\\
                 \glt `\ldots if you and Mary want to.'}
            \ex {\gll \ldots wenn($^*$-st)\slash($^*$-ts) [die Maria und du] woll-ts\slash-en.\\
                      \ldots if-\g{2.sg}\slash$^*$-\g{2.pl} [the Mary and you.\g{sg}] want-\g{2.pl}\slash-\g{3.pl}\\}
        \end{xlist}
        \ex
        \begin{xlist}
            \ex {\gll \ldots wenn-ts [es und die Maria] woll-ts\slash-en.\\
                      \ldots if-\g{2.pl} [you.\g{pl} and the Mary] want-\g{2.pl}\slash-\g{3.pl}\\
                 \glt `\ldots if you and Mary want to.'}
            \ex {\gll \ldots wenn($^*$-ts) [die Maria und es] woll-ts\slash-en.\\
                      \ldots if-\g{2.sg}\slash$^*$-\g{2.pl} [the Mary and you.\g{pl}] want-\g{2.pl}\slash$^*$-\g{3.pl}\\}
        \end{xlist}
        \ex
        \begin{xlist}
            \ex {\gll \ldots wenn($^*$-ts) [ihr und die Maria] woll-ts\slash$^*$-en.\\
                      \ldots if-\g{2.pl} [you.\g{pl} and the Mary] want-\g{2.pl}\slash$^*$-\g{3.pl}\\
                 \glt `\ldots if you and Mary want to.'}
            \ex {\gll \ldots wenn($^*$-ts) [die Maria und ihr] woll-ts\slash$^*$-en.\\
                      \ldots if-\g{2.sg}\slash$^*$-\g{2.pl} [the Mary and you.\g{pl}] want-\g{2.pl}\slash$^*$-\g{3.pl}\\}
        \end{xlist}
    \end{exe}

    Agreement can also operate across intervening adverbs and dative arguments:
    %
    \begin{exe}
        \ex {\gll Wenn-st gestern dem Hans (du) derart deppert kumm-st, brauch-st (du) di net wundern dass er di heut verarscht.\\
                  If-\g{2.sg} yesterday the.\g{dat} Hans (you.\g{sg}) this stupid come-\g{2.sg}, need-\g{2.sg} (you.\g{sg}) SE not wonder that he you today mocks\\
             \glt `If you acted like an idiot towards Hans yesterday, you shouldn't be surprised that he's mocking you today.'}
    \end{exe}

    Provide two finite-state automata, one for subject-verb agreement, and one for complementizer-subject agreement.
    Together, they should correctly account for the contrasts noted above.
    Then construct an automaton that enforces both agreement patterns.
    Are the regularities of the original two automata easy to discern in this automaton?
\end{exercise}

\begin{exercise}
    Icelandic allows long-distance agreement between subjects and the finite verb.
    %
    \begin{exe}
        \ex {\gll Það voru konugi gefnar ambátt-ir í vettur.\\
                  there were king.\g{dat} given slaves.\g{nom} in winter\\
             \glt `Slaves were given to the king in the winter.}
    \end{exe}
    %
    Surprisingly, long-distance agreement is optional in bi-clausal constructions.
    %
    \begin{exe}
        \ex
        \begin{xlist}
            \ex {\gll Einhverjum stúdent finnst tölvurnar ljótar.\\
                      Some.\g{dat} student.\g{dat} finds.\g{sg} computers.\g{def.nom} ugly.\g{nom}\\}
            \ex {\gll Einhverjum stúdent finnast tölvurnar ljótar.\\
                      Some.\g{dat} student.\g{dat} finds.\g{pl} computers.\g{def.nom} ugly.\g{nom}\\
                 \glt `Some student finds the computers ugly.'}
        \end{xlist}
    \end{exe}
    %
    And to complicate matters even further, a linearly intervening dative argument blocks long-distance agreement.
    %
    \begin{exe}
        \ex
        \begin{xlist}
            \ex[]  {\gll Það virðist einhverjum manni hestarnir vera seinir.\\
                        There seems.\g{sg} some.\g{dat} man.\g{dat} horses.\g{def.nom} be slow.\g{nom}\\}
            \ex[*] {\gll Það virðast einhverjum manni hestarnir vera seinir.\\
                        There seems.\g{pl} some.\g{dat} man.\g{dat} horses.\g{def.nom} be slow.\g{nom}\\
                    \glt `The horses seem slow to some man.'}
        \end{xlist}
    \end{exe}
    %
    Construct a finite-state automaton that captures the contrast above.
    To simplify things, you may assume that every string contains exactly one finite verb and that nominative and dative arguments can be unambiguously identified based on their case suffix.
\end{exercise}

\begin{exercise}
    At an abstract level, inflected complementizers in Bavarian German and Icelandic long-distance agreement seem very similar in that they both can operate locally or at a distance, and are sometimes optional.
    Only the latter seems to be subject to a robust intervention effect, however.
    Are these similarities and discrepancies readily apparent from the automata you drew in the previous exercises?
\end{exercise}
