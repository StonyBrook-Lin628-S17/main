\chapter{Meta-Rules and Tree Automata}
\label{cha:metarules}

Do all exercises except 5.4, which is optional.

\begin{exercise}
    German has a three-way agreement system for adjectives.
    If an adjective is used predicatively, it always surfaces in its base form without any inflectional affixes.
    %
    \begin{exe}
        \ex
        \begin{xlist}
            \ex \gll Ein\slash Der Bub ist hungrig.\\
                     a\slash the.\textsc{Masc} boy is hungry\\
            \ex \gll Eine\slash Die Frau ist hungrig.\\
                     a\slash the.\textsc{Fem} woman is hungry\\
            \ex \gll Ein\slash Das Tier ist hungrig.\\
                     a\slash the.\textsc{Neut} animal is hungry\\
            \ex \gll Die Buben\slash Frauen\slash Tiere sind hungrig.\\
                     The.\textsc{pl} boys\slash women\slash animals are hungry\\
        \end{xlist}
    \end{exe}
    %
    If the adjective is NP-internal, agreement depends on the form of the determiner.
    With a definite determiner, agreement only indicates number.
    With an indefinite determiner, agreement is in number and gender.
    %
    \begin{exe}
        \ex
        \begin{xlist}
            \ex der hungrig-e Bub
            \ex die hungrig-e Frau
            \ex das hungrig-e Tier
            \ex die hungrig-en Buben\slash Frauen\slash Tiere
        \end{xlist}
        \ex
        \begin{xlist}
            \ex ein hungrig-er Bub
            \ex eine hungrig-e Frau
            \ex ein hungrig-es Tier
            \ex $\emptyset$ hungrige-e Buben\slash Frauen\slash Tiere
        \end{xlist}
    \end{exe}
    %
    Write a context-free grammar (without meta-rules!) that correctly handles these adjectival agreement patterns.
    Try to make your grammar as small and compact as possible.
\end{exercise}

\begin{exercise}
    Continuing the previous exercise, write an equivalent grammar that uses meta-rules (or even meta-meta rules, if you can find a use for them).
    Is your grammar significantly more succinct than the previous grammar?
    Why (not)?
\end{exercise}

\begin{exercise}
    Still continuing the previous exercise, modify both grammars so that they also accommodate the fact that NP-internal adjectives display case agreement as indicated below.
    %
    \begin{center}
        \begin{tabular}{rcccccc}
            \toprule
            \textbf{Case} & \textbf{m indef} & \textbf{m def} & \textbf{f indef} & \textbf{f def} & \textbf{n indef} & \textbf{n def}\\
            \midrule
            Nom Sg & -er & -e  & -e  & -e  & -es & -e  \\
            Gen Sg & -en & -en & -en & -en & -en & -en \\
            Dat Sg & -en & -en & -en & -en & -en & -en \\
            Acc Sg & -en & -en & -e  & -e  & -es & -e  \\
            \midrule
            Nom Pl & -e  & -en & -e  & -en & -e  & -en \\
            Gen Pl & -er & -en & -er & -en & -er & -en \\
            Dat Pl & -en & -en & -en & -en & -en & -en \\
            Acc Pl & -e  & -en & -e  & -en & -e  & -en \\
            \bottomrule
        \end{tabular}
    \end{center}
    %
    Which one of the two grammars is easier to expand accordingly?
\end{exercise}

\begin{exercise}
    Try writing a bottom-up tree automaton that checks agreement for NP-internal adjectives.
    Is this easier or harder than writing a grammar with meta-rules?
\end{exercise}

\begin{exercise}
    In Icelandic, adjectives in predicative position display resolved agreement if the subject is a coordination of multiple DPs:
    %
    \begin{enumerate}
        \item If all conjuncts are masculine, the adjective displays masculine agreement.
        \item If all conjuncts are feminine, the adjective displays feminine agreement.
        \item In all other cases, the adjective displays neuter agreement.
    \end{enumerate}
    %
    Write a tree automaton that correctly assigns to each DP its resolved gender.
    For example, a DP that contains a single masculine noun should be assigned a state for masculine gender, a DP that is a coordination of two feminine DPs should receive feminine, and a DP that is a coordination of a masculine and a feminine noun receives the state for neuter gender.
    You may assume that all coordinations have a binary branching tree structure:
    %
    \begin{center}
        \begin{forest}
            [DP
                [some DP]
                [D$'$
                    [D
                        [and]
                    ]
                    [some other DP]
                ]
            ]
        \end{forest}
    \end{center}
\end{exercise}

\begin{exercise}
    This continues the previous exercise.
    A complicating factor is that Icelandic has \emph{hybrid nouns}, i.e.\ nouns whose syntactic gender differs from their semantic gender.
    The best-known example is \emph{hetja} `hero', which is syntactically feminine but usually refers to a man and thus is semantically masculine.
    Another one is \emph{skáld} `poet', which is syntactically neuter but may refer to a man or a woman.
    Do you need to modify the automaton from the previous exercise to accommodate hybrid agreement?
    If so, what changes are necessary?
\end{exercise}
