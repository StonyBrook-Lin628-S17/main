\chapter{The Loop Hole of Subcategorization}
\label{cha:subcategorization}

You have to do exercise 7.1, 7.2, and 7.6.
Do the others until you feel you got the point.


These exercises use the following feature notation:
%
\begin{center}
    \begin{tabular}{rl}
        \fcat{X} & category X\\
        \fsel{X} & select an X\\
        \flcr{f} & landing site for f-movement\\
        \flce{f} & undergoes f-movement
    \end{tabular}
\end{center}

\begin{exercise}
    Draw all the derivation trees that are licensed by the following grammar:

    \begin{center}
        \begin{tabular}{ll}
            \mlex{the}{\fsel{N} \fcat{D}}
            &
            \mlex{woman}{\fcat{N}}
            \\
            \mlex{the}{\fsel{N} \fcat{D} \flce{nom}}
            &
            \mlex{women}{\fcat{N}}
            \\
            \mlex{the}{\fsel{N} \fcat{D} \flce{top}}
            &
            \\
            &
            \mlex{like}{\fsel{D} \fsel{D} \fcat{V}}
            \\
            \mlex{which}{\fsel{N} \fcat{D} \flce{wh}}
            &
            \mlex{likes}{\fsel{D} \fsel{D} \fcat{V}}
            \\
            \mlex{which}{\fsel{N} \fcat{D} \flce{nom} \flce{wh}}
            &
            \mlex{liked}{\fsel{D} \fsel{D} \fcat{V}}
            \\
            &
            \mlex{$\emptystring$}{\fsel{V} \flcr{nom} \fcat{T}}
            \\
            \mlex{$\emptystring$}{\fsel{T} \fcat{C}}
            &
            \\
            \mlex{$\emptystring$}{\fsel{T} \flcr{top} \fcat{C}}
            &
            \\
            \mlex{$\emptystring$}{\fsel{T} \flcr{wh} \fcat{C}}
            &
            \\
            \mlex{do}{\fsel{T} \flcr{wh} \fcat{C}}
            &
            \\
            \mlex{does}{\fsel{T} \flcr{wh} \fcat{C}}
            &
        \end{tabular}
    \end{center}

    Indicate for each derivation tree whether it is well-formed for English.
\end{exercise}

\begin{exercise}
    Consider the fragment of the grammar where only subject movement from Spec,VP takes place.
    Write a tree automaton that enforces subject verb agreement.
    Compile it into the grammar.
\end{exercise}

\begin{exercise}
    Now expand your set such that topicalization is also allowed, and movement may also take place from the object position.
    Adapt your automaton so that it forbids subject movement from any position but Spec,VP, while still ensuring subject verb agreement.
    Compile it into the grammar.
\end{exercise}

\begin{exercise}
    Now consider the fragment of the grammar where wh-movement always takes place, either from the specifier or the complement of V.
    Assume that subject movement always originates in Spec,VP.
    Write a tree automaton that ensures subject verb agreement in this case, and compile it directly into the grammar.
\end{exercise}

\begin{exercise}
    Only for those with lots of time on their hand: take the full grammar and refine it so that it only allows for well-formed derivations of English.
\end{exercise}

\begin{exercise}
    If the grammar also contains the lexical item \mlex{which}{\fsel{N} \fcat{D}} it becomes possible to generate echo questions and sentences like \emph{who likes what}, but we also ungrammatical sentences like \emph{what does who like}.
    Write a tree automaton that implements Relativized Minimality.
\end{exercise}
